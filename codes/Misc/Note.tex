開始寫題目之前,請做下面的事:
\begin{itemize}
    \item 在「開始寫任何題目之前」,應該要先自己看過「所有」範例測資的邏輯和演算法是否有跟範例輸出對上
    \item 如果你覺得別人的某段程式碼有錯誤,就應該直接講出來
    \item +2~+3 後就開始生測資跟對拍(根據寫 generator 跟 checker 的時間決定)
\end{itemize}

寫程式請遵照以下原則:
\begin{itemize}
    \item 準確使用註解分段程式碼
    \begin{itemize}
        \item declare
        \item init
        \item input
        \item process / queries
        \item output
    \end{itemize}
    \item 陣列若可以開到最大,則使用常數宣告大小
\end{itemize}

上傳之前,請依序檢查以下資訊:
\begin{enumerate}
    \item 是否開啟 IO 優化
    \item 是否有 t 筆輸入但忘了輸入
    \item 是否有初始化容器
    \item 題目範圍有沒有開到最大
    \item 跑過所有範例測資,並嚴格確認是否正確
\end{enumerate}